\documentclass{beamer}
\usepackage{amsmath, amssymb, amsthm}
\usepackage{braket}
\usepackage{algorithm}
\usepackage{algpseudocode}
\newcommand{\sgn}{\operatorname{sgn}}
\newcommand{\norm}[1]{\left\lVert#1\right\rVert}
\usepackage{etoolbox}
\newcommand{\abs}[1]{\lvert#1\rvert}
\newcommand{\vect}[1]{\left(#1\right)}

\usetheme{Copenhagen}

\title{Round Efficient Secure Multiparty Quantum Computation with Identifiable Abort}

\author[Alon, Chung, Chung, Huang, Lee, Shen]{Bar Alon \inst{1} \and Hao Chung \inst{2} \and Kai-Min Chung \inst{3} \and Mi-Ying Huang \inst{3} \and \underline{Yi Lee} \inst{3} \and Yu-Ching Shen \inst{3}}
\institute{\inst{1} Ariel University \and \inst{2} Carnegie Mellon University \and \inst{3} Academia Sinica}
\date{IACR Crypto 2021}

\begin{document}

\frame{
	\titlepage
}

\frame{
	\frametitle{Defining the Problem}
	\begin{itemize}[<+->]
		\item Goal: Among $n$ parties, compute some fixed general functionality
			$$f\vect{x_1, \ldots, x_n} \mapsto \vect{y_1, \ldots, y_n}$$
		\item $x_i$ and $y_i$ are the \emph{private} input and output for party $i$.
		\item We consider the \emph{fully} quantum setting. $X$ and $Y$ are qubits, and $f$ is an efficient CPTP.
		\item We achieve security with \emph{identifiable} abort: If the protocol aborts, the honest parties agree on the identity of a malicious party.
	\end{itemize}
}

\frame{
	\frametitle{Prior works}
	\begin{itemize}[<+->]
		\item Study of MPC dates back to Yao's paper in FOCS '86.
		\item SWIA was formulated in CRYPTO '14 paper by Ishai, Ostrovsky, and Zikas. There are earlier protocols known to also admit SWIA.
		\item First MPQC was constructed in EUROCRYPT '20 paper by Dulek et al. Many ideas from this paper are reused in our construction.
	\end{itemize}
}

\frame{
	\frametitle{Security with Identifiable Abort: Challenges}
	\begin{itemize}[<+->]
		\item Consider a subproblem: Alice sends some quantum state $\rho$ to Bob.
		\item What if the message is ``dropped"?
		\item By no-cloning theorem, $\rho$ is now irrecoverable.
		\item Alice and Bob accuse each others, other parties can't tell who's lying.
	\end{itemize}
}

\frame{
	\frametitle{Our solution: Routing}
	\begin{itemize}[<+->]
		\item Alice protects her qubit using error-correcting code.
		\item Define a \emph{Trustfulness Graph} $G=\vect{V, E}$.
			\begin{itemize}[<+->]
				\item $V=TODO$
				\item $E=TODO$
			\end{itemize}
		\item ``Route" qubits along the graph...
		\item If a packet is dropped, the corresponding edge ``goes down".
		\item Compose routing with Clifford authentication code to get \emph{Authenticated Routing}.
	\end{itemize}
}

\frame{
	\frametitle{Are we done?}
	\begin{itemize}[<+->]
		\item So, compose Authenticated Routing and homomorphic QECCs. We're done, right?
		\item What if adversary provides invalid QECC codewords?
		\item We need to apply QECC decryption before computation.
		\item Qubits can't be dropped after QECC decryption.
		\item Computation instead by reducing to a \emph{verifiable homomorphic encryption}.
	\end{itemize}
}

\frame{
	\frametitle{Future Directions}
	\begin{itemize}[<+->]
		\item Constant-round impossibility?
		\item Universal Composability?
		\item Adaptive corruptions?
		\item Require more adversaries to cause an abort?
	\end{itemize}
}

\end{document}
